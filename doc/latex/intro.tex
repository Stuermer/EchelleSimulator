Echelle spectrographs are named after its main optical component -\/ an {\itshape Echelle} {\itshape grating} (from French échelle). They are used to obtain high-\/resolution, cross-\/dispersed spectra and are widely used in astronomy.

Echelle gratings have a large ruling spacing and are used in high diffraction order allowing for a high spectral resolution. Spectra of different diffraction orders are overlapping and a second component -\/ a so called {\itshape cross} {\itshape disperser} -\/ is used to seperate the individual orders. Compared to an echelle grating, the dispersion of the cross disperser is low and a typical choice is a low dispersion grating, a prism or a grism.

On the whole, an echelle spectrograph produces a 2D spectrum, where the positions of a certain wavelength depends on details of the spectrograph optics, its echelle parameters and its cross disperser.

 